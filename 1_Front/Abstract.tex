\chapter*{Abstract}
\phantomsection % Add this for problems with referring in TOC ?
\addcontentsline{toc}{chapter}{Abstract}

Class-D amplifiers are commonly used in audio amplifications to achieve a highly efficient power amplification stage. The topic of this project is to investigate the parasitic elements of the output filter and its effects on the system. In transient states of a switch-mode audio amplifier and the complementary filtering stage, aliasing can be introduced. When this aliasing is fed back through the regulator loop it becomes part of the input signal. The consequence of this can be that the amplified output signal has a significant component of the introduced aliasing signal, which can lead to distortion in the signal that is put to the loudspeaker. In order to gain a sufficient understanding of the design of a class-D audio amplifier, one will be designed and built. Afterwards, some analysis revealed a significant change in the control loop in a specific frequency band from the introduction of parasitic elements in the inductor of the output filter. Through a second synthesis calculation it was proposed to make the integrator element in the control loop more aggressive to compensate for influence of parasitic elements in a higher frequency band.



