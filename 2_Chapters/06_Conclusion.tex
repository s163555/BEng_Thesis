\chapter[Conclusion]{Conclusion}
The overall goal of this project was to work on a class-D audio amplifier system, understand its topology and seek to investigate the influence of parasitic elements on the control loop. A class-D audio amplifier design was researched from a reference and implemented in a Spice simulation. The simulation was subject to some rudimentary analysis to better learn the functionality of the design. The output filter was subject to a small-signal analysis to model the frequency response of parasitic elements in both the inductor and capacitor. Next, a hardware implementation was built using calculated component values. The hardware implementation was applied in a parameterised testing scheme to investigate the influence of the output filter on each control loop transfer function. Validation was performed on the measurement data from a theoretical standpoint. The investigation revealed inductor parasitic elements primarily affected the outer control loop above \SI{10}{\kilo\hertz} but lower than \SI{100}{\kilo\hertz}. It would appear the LQR control loop effect on introduced parasitic elements was negligible. The LQR control loop did seem to effectively mitigate discontinuity in the frequency range in the vicinity of the modulation frequency in both magnitude and phase.  Based on these findings, a proposition was considered to decrease the time-constant of the PI regulator to compensate for the parasitic elements by employing a more aggressive integrative effect on the PI controller. 

\section[Future Work]{Future Work}
Regarding the conclusion of the investigation into parasitic elements of the capacitor, the logical next step would be to perform the proposed control loop modifications and redo the control loop analysis to see if the outer control loop has compensated for the influences of the parasitic elements. This was not possible to do due to time constraints.
Secondly, it was not possible to investigate the parasitic elements of the output filter capacitor. On the topic of this project, the series inductance and series resistance would be the following subject to investigate. Regrettably, due to the form factor of the hardware, it was a laborious task to properly mount the parasitic elements in series with the output filter capacitor. Therefore, it was promptly decided to limit the scope of the investigation to the inductor parasitic elements. This is ideal for a subsequent investigation to broaden the scope of model parameters to comprehensively examine the serial resistance and inductance of the output filter capacitor.
Another case for future investigation could be more thoroughly account for the inner control loop (LQR) in regards to parasitic elements. It is conceivable this can be investigated further to include parasitic elements in the state space model when designing the parameters of the inner control loop.